\chapter{Programming}
\mylabel{sec:program}

\noindent
\nip{} includes a tiny lazy functional programming language. You can use it to
glue VIPS image processing functions together to perform more complicated
tasks.  All of the \nip{} toolkit menus are written in this language.

These first sections just describe the programming language. See
\pref{sec:progwin} for a description of the programming window.  You use
\nip{}'s programming language to control the user interface: the link between
what happens inside a \nip{} function and what you see on the screen in
covered in \pref{sec:bowser}.

\section{Load and save}

When \nip{} starts up it loads all of the definition files (files with a
\ct{.def} extension) it can find in the directories listed in your start
path. You can change the start path in Preferences. By default, the start
path lists just two areas: a personal start directory that \nip{} makes in
your home area, and the main system \nip{} start directory containing all
the standard toolkits.

If there are two files with the same name on the start path, then
\nip{} will only load the first one. This means that if you modify one
of \nip{}'s built-in menus and save it to your personal start directory,
in future you'll just see your personalised version. 

You can load or reload a toolkit at any time with the \ctr{File}\ct{Open
Toolkit} menu item in the program window. If you open a toolkit with the same
name as an existing toolkit, \nip{} will remove the old toolkit before it
loads the new one. 

\section{Using an external editor}

If you're going to be doing any more than a little programming in \nip{} you
probably won't want to use the built-in editor. I suggest you start your
favorite editor in one window on the screen and then in the \nip{} program
window click \ctr{File}\ct{Open Toolkit} and check the Pin-up box in the file
selector.

Now every time you want to try out your definition, save the file from your
external editor and click OK in \nip{}'s file selector. 

\nip{}'s editor automatically adds some semicolon characters to separate
definitions in a file. If you're using an external editor, you'll need to put
these in yourself. Also check the syntax for adding separators and column
items to menus.

\section{Syntax}

The most basic sort of definition looks like this:

\begin{verbatim}
// very simple!
fred = 12
\end{verbatim}

\noindent
This defines a function called \ct{fred} whose value is the number 12. The
\ct{//} marks a comment: everything to the end of the line is skipped.
Case is distinguished, so \ct{Fred} and \ct{fred} are two different functions.
You can use letters, numbers, underscores and single quotes in function names.

You can have patterns on the left of the equals sign. For example:

\begin{verbatim}
[fred, petra] = [12, 13]
\end{verbatim}

\noindent
defines \ct{fred} to have the value 12 and \ct{petra} to have the value 13.
See \pref{sec:pattern} for details.

Functions may take parameters:

\begin{verbatim}
/* A function with parameters.
 */
jim a b = a + b + 12
\end{verbatim}

\noindent
This defines a function called \ct{jim} which takes two parameters and whose
value is the sum of the two parameters, plus 12. The \ct{/*} and \ct{*/}
enclose a multi-line comment.

Functions may have several right-hand-sides, each right-hand-side qualified by
a guard expression. Guards are tested from top to bottom and the first guard
which has the value \ct{true} causes the function to have the value of that
right-hand-side. If no guard evaluates to \ct{true}, then the last
right-hand-side is used.

\begin{verbatim}
jenny a b
  = 42, a + b >= 100
  = 43, a + b >= 50
  = 44
\end{verbatim}

\noindent
This defines a function called \ct{jenny} which takes two parameters and whose
value is 42 if the sum of the parameters is 100 or greater; 43 if the sum is
greater than or equal to 50 but less than 100; and 44 if the sum is less than
50.

Any function may be followed by any number of local functions, enclosed in
curly braces. So \ct{jenny} could be written as:

\begin{verbatim}
jenny a b
  = 42, sum >= 100
  = 43, sum >= 50
  = 44
{
  sum = a + b;
}
\end{verbatim}

\noindent
Note that you need a semi-colon after each local function.  A local function
may refer to anything in an enclosing scope, including itself.

You can write \ct{if-then-else} expressions:

\begin{verbatim}
david a = if a < 12 then "my cat"
  else "likes lasagne"
\end{verbatim}

\noindent
This is exactly equivalent to:

\begin{verbatim}
david a
  = "my cat", a < 12
  = "likes lasagne"
\end{verbatim}

\noindent
\ct{if-then-else} expressions are sometimes easier to read than guards.

Functions application is with spaces (juxtaposition). For example:

\begin{verbatim}
harry = jim 2 3
\end{verbatim}

\noindent
defines \ct{harry} to have the value 17. 

All functions are curried, that is, they can accept their arguments in
stages. For example:

\begin{verbatim}
sandro = jim 1
\end{verbatim}

\noindent
defines \ct{sandro}, a function which takes one parameter and will add 13 to
it. This trick becomes very useful with list processing, see \pref{sec:lists}.

\nip{} has some built-in functions, see \tref{tb:builtin}. They mostly 
take a single argument. All other functions are defined in the various
standard toolkits and can be edited in the program window.

\begin{tab2}
\begin{center}
\begin{tabular}{||l|l||}
\hline
Function & Description \\
\hline
\ct{dir} \textit{any}		& List names in scope \\
\ct{has\_member} \textit{[char]} \textit{any}	& 
				Does class have member \\
\hline
\ct{name2gtype} \textit{[char]}	& Search for a GType by name \\
\ct{gtype2name} \textit{real}	& Return the name of a GType \\
\hline
\ct{error} \textit{[char]}	& Stop with error message \\
\ct{print} \textit{any}		& Convert to string \\
\ct{expand} \textit{[char]}	& Expand environment variables in string \\
\ct{search} \textit{[char]}	& Search for a file \\
\ct{\_} \textit{[char]}		& Translate string \\
\hline
\ct{is\_image} \textit{any}	& Test for image \\
\ct{is\_bool} \textit{any}	& Test for boolean \\
\ct{is\_real} \textit{any}	& Test for real	\\
\ct{is\_class} \textit{any}	& Test for class \\
\ct{is\_char} \textit{any}	& Test for char	\\
\ct{is\_list} \textit{any}	& Test for list	\\
\ct{is\_complex} \textit{any}	& Test for complex \\
\ct{is\_instanceof} \textit{[char]} \textit{any}	& 
				Test for instance of class	\\
\hline
\ct{re} \textit{image}/\textit{complex}/\textit{class} & 
				Extract real part of complex	\\
\ct{im}	\textit{image}/\textit{complex}/\textit{class} &
				Extract imaginary part of complex \\
\ct{hd} \textit{list}		& Extract head of list \\
\ct{tl}	\textit{list}		& Extract tail of list \\

\ct{sin} \textit{image}/\textit{number}/\textit{class} & 
				Sine \\
\ct{cos} \textit{image}/\textit{number}/\textit{class} & 
				Cosine \\
\ct{tan} \textit{image}/\textit{number}/\textit{class} & 
				Tangent \\
\ct{asin} \textit{image}/\textit{number}/\textit{class} & 
				Arc sine \\
\ct{acos} \textit{image}/\textit{number}/\textit{class} & 
				Arc cosine \\
\ct{atan} \textit{image}/\textit{number}/\textit{class} & 
				Arc tangent \\
\ct{log} \textit{image}/\textit{number}/\textit{class} & 
				Natural log \\
\ct{log10} \textit{image}/\textit{number}/\textit{class} & 
				Base 10 log \\
\ct{exp} \textit{image}/\textit{number}/\textit{class} & 
				e to the power \\
\ct{exp10} \textit{image}/\textit{number}/\textit{class} & 
				10 to the power \\
\ct{ceil} \textit{image}/\textit{number}/\textit{class} & 
				Round up \\
\ct{floor} \textit{image}/\textit{number}/\textit{class} & 
				Round down \\
\ct{gammq} \textit{real} \textit{real} & 
				Normalised incomplete Gamma function \\
\hline
\ct{vips\_image} \textit{[char]}& Load image from file \\
\ct{read} \textit{[char]}	& Load file as a string \\
\hline
\end{tabular}
\end{center}
\caption{\nip{} built in functions}
\mylabel{tb:builtin}
\end{tab2}

\section{Naming conventions}

You can name things in any way you like, but we've used the following
conventions.

\begin{itemize}

\item
Classes start with a capital letter, words are separated with underscores,
subsequent words are not capitalised (eg. \ct{Image\_file})

\item
Private names are prefixed with underscores (and are hidden by most of the
user interface)

\item
Functions from the VIPS library are prefixed with \ct{im\_}

\item
Global utility functions (eg. \ct{map}), public members (eg.
\ct{Colour.colour\_space}) are all lower case, words are separated
with underscores, subsequent words are not capitalised

\item
Constants are capitalised (eg. \ct{Operator\_type.COMPOUND\_REWRAP})

\end{itemize}

\section{Evaluation}

\nip{} calculates the value of an expression by using the definitions you
entered to successively reduce the expression until it becomes one of the
base types. Sometimes there is a choice as to which part of the expression
will be reduced next --- \nip{} will always choose to reduce the leftmost,
outermost part of the expression first.

For example, consider this definition:

\begin{verbatim}
factorial n
  = n * factorial (n - 1), n > 1
  = 1
\end{verbatim}

And here's how \nip{} will evaluate the expression \ct{factorial 3}:

\begin{verbatim}
factorial 3 -->
  3 > 1 -->
  true
3 * factorial (3 - 1) -->
  (3 - 1) > 1 -->
  2 > 1 -->
  true
3 * (2 * factorial (2 - 1)) -->
  (2 - 1) > 1 -->
  1 > 1 -->
  false
3 * (2 * 1) -->
3 * 2 -->
6
\end{verbatim}

\noindent
Note how \nip{} delays evaluating parameters to functions until they are
needed, but still shares the result. \ct{3 - 1} is only evaluated once,
for example, even though the result is used three times.  \nip{} has a trace
window: click on \ctr{Debug}\ct{Trace} in the program window and check the
\ctr{View}\ct{Operators} menu item.

The advantage of this style of computation over conventional imperative
programming languages is that you can reason about your program
mathematically\footnote{Since programs are referentially transparent (that
is, the value of an expression depends only upon its syntactic context,
not upon computation history), you can easily do equational reasoning,
proof by induction, and so on.

Expressions are like theorems, definitions are like axioms, computation
is like proof.}. 

This isn't the best way to write a factorial function. A function with lots
of recursive calls can be hard to understand --- it's much better to use one
of the higher order functions from the standard environment to encapsulate
the type of recursion you want to use.

The clearest definition for factorial is probably:

\begin{verbatim}
factorial n = product [1..n]
\end{verbatim}

\noindent
See \pref{sec:listsyntax} for an explanation of the list syntax.

\section{Operators}
\mylabel{sec:operators}

\nip{}'s expression syntax is almost exactly the same as C,
with a few small changes. \tref{tb:precedence} lists all of 
\nip{}'s operators in order of increasing precedence. If you've used C, the
differences are:

\begin{itemize}

\item
	C's \verb+?:+ operator becomes \ct{if-then-else}, see above

\item
	Like almost every functional language, \nip{} uses square brackets
	for list constants (see \pref{sec:listsyntax}), so to index a
	list, \nip{} uses \ct{?}

\item
	\nip{} adds \ct{@} for function composition, see \pref{sec:func}

\item
	The \ct{:} operator is infix list cons, see \pref{sec:lists}

\item
	The \ct{++} operator becomes an infix concatenation operator, \ct{--}
	becomes list difference. Again, see \pref{sec:listsyntax}

\end{itemize}

The only slightly tricky point is that function application binds very
tightly (only list index and class project bind more tightly). So the
expression:

\begin{verbatim}
jim = fred 2 + 3
\end{verbatim}

\noindent
binds as:

\begin{verbatim}
jim = (fred 2) + 3
\end{verbatim}

\noindent
This is almost always the behaviour you want.

There are two special equality tests: \ct{===} and 
\ct{!==}. These test for pointer equality, that is, they return \ct{true} if
their arguments refer to the same object. These are occasionally useful for
writing interactive functions.

\begin{tab2}
\begin{center}
\begin{tabular}{||l|l|l||}
\hline
Operator & Associativity & Description \\
\hline
\ct{if then else}	& Right			& If-then-else construct \\
\ct{=>}			& Left			& Form name/value pair \\
\verb+||+		& Left			& Logical or \\
\ct{\&\&}		& Left			& Logical and \\
\ct{@}			& 			& Function composition 
						  (see \pref{sec:func}) \\
\verb+|+		& Left			& Bitwise or \\
\rtp{}			& Left			& Bitwise exclusive or \\
\ct{\&}			& Left			& Bitwise and \\
\hline
\ct{==}			& Left			& Equal to\\
\ct{!=}			& 			& Not equal to\\
\ct{===}		& 			& Pointer equal to\\
\ct{!==}		& 			& Pointer not equal to\\
\hline
\ct{<}			& Left			& Less than \\
\ct{<=}			& 			& Less than or equal to\\
\ct{>}			& 			& Greater than \\
\ct{>=}			& 			& Greater than or equal to\\
\hline
\ct{<<}			& Left			& Left shift \\
\ct{>>}			& 			& Right shift \\
\hline
\ct{+}			& Left			& Addition \\
\ct{-}			& 			& Subtraction \\
\ct{*}			& Left			& Multiplication \\
$/$			& 			& Division \\
\ct{\%}			& 			& Remainder after division \\
\ct{!}			& Left			& Logical negation \\
\verb+~+		& 			& One's complement \\
\ct{++}			& 			& Join 
						  (see \pref{sec:listsyntax}) \\
\verb+--+		& 			& Difference 
						  (see \pref{sec:listsyntax}) \\
\ct{-}			& 			& Unary minus \\
\ct{+}			& 			& Unary plus \\
\ct{(}\emph{type}\ct{)}	& 			& Type cast expression \\
\ct{**}			& Right			& Raise to power \\
\ct{:}			& 			& List CONS
						  (see \pref{sec:listsyntax}) \\
\emph{space}		& Left			& Function application \\
\ct{?}			& Left			& List index 
						  (see \pref{sec:listsyntax}) \\
\ct{.}			& Left			& Class project 
						  (see \pref{sec:class}) \\
\hline
\end{tabular}
\end{center}
\caption{\nip{} operators in order of increasing precedence}
\mylabel{tb:precedence}
\end{tab2}

\subsection{The real type}

\nip{} has a single number type for integers and real numbers. All are
represented internally as 64-bit floating point values. You can use the four
standard arithmetic operators (\ct{+}, \ct{-}, \ct{*}, \ct{/}), remainder
after integer division (\%), raise-to-power (\ct{**}), the relational
operators (\ct{<}, \ct{<=}, \ct{>}, \ct{>=}, \ct{==}), the bitwise logical
operators (\ct{\&}, \verb+|+, \rtp{}, \verb+~+), integer shift operators
(\ct{<<}, \ct{>>}) and unary negation and positive (\ct{-}, \ct{+}).

Other mathematical functions are pre-defined for you:  \ct{sin}, \ct{cos},
\ct{tan}, \ct{asin}, \ct{acos}, \ct{atan}, \ct{log}, \ct{log10}, \ct{exp},
\ct{exp10}, \ct{ceil}, \ct{floor}. Each has the standard behaviour.

You can use type-casts on reals. However, they remain 64-bit floating point,
the range is simply clipped. Casting to \ct{unsigned short} produces a 64-bit
float whose fractional part has been set to zero, and which has been
clipped to the range 0 to 65535. This may or may not cause rounding problems.

You can write hexadecimal number constants as \verb"0xff".

\subsection{The complex type}

Complex numbers are rather sketchily implemented. They are generally handy for
representing vectors and coordinates rather than for doing arithmetic, so the
range of operations is limited.

Complex constants are written as two numbers enclosed in round brackets and
separated by a comma. You can use the four standard arithmetic operators
(\ct{+}, \ct{-}, \ct{*}, \ct{/}), raise-to-power (\ct{**}), and unary negation
and positive (\ct{-}, \ct{+}). You can use \ct{==} only of the relational
operators. You can mix complex and real numbers in expressions. You can cast
reals to complex and back.  Use the functions \ct{re} and \ct{im}
to extract the real and imaginary parts.

\begin{verbatim}
(12, 13) + 4 == (16, 13)
(12, 2 + 2) ==  (12, 4)
re (12, 13) == 12
im (12, 13) == 13
\end{verbatim}

\subsection{The character type}

Character constants are written as single characters enclosed in single
quotes. You can use the relational operators (\ct{<}, \ct{<=}, \ct{>},
\ct{>=}, \ct{==}) to sort characters by ASCII order. You can cast a
character to a real to get its ASCII value. You can cast a real ASCII value 
to a character. You can use the standard C escapes to represent non-ASCII
characters.

\begin{verbatim}
(int) 'A' == 65
(char) 65 == 'A'
is_digit x = '0' <= x && x <= '9'
newline == '\n' 
\end{verbatim}

\subsection{The boolean type}

The two boolean constants are written as \ct{true} and \ct{false}. Boolean
values are generated by the relational operators.  You can use the standard
logical operators (\ct{\&\&}, \verb+||+, \ct{!}). You can use a boolean type
as an argument in an \ct{if-then-else} expression.

As with C, the logical operators do not evaluate their right-hand sides if
their value can be determined just from evaluating their left-hand sides.

\begin{verbatim}
true && false == false
true || error "boink!" == true
if true then 12 else 13 == 12
\end{verbatim}

\subsection{The list type}
\mylabel{sec:listsyntax}

Lists are created from two constructors. \ct{[]} denotes the empty list. The
list construction operator (\ct{:}, pronounced CONS by LISP programmers) takes
an item and a list, and returns a new list with the item added to the front.
As a convenience, \nip{} has a syntax for list constants. A list constant is a
list of items, separated by commas, and enclosed in square brackets:

\begin{verbatim}
12:[] == [12]
12:13:14:[] == 12:(13:(14:[])) == 
  [12,13,14]
[a+2,3,4] == (a+2):3:4:[]
[2]:[3,4] == [[2],3,4]
\end{verbatim}

Use the functions \ct{hd} and \ct{tl} to take the head and the
tail of a list:

\begin{verbatim}
hd [12,13,14] == 12
tl [12,13,14] == [13,14]
\end{verbatim}

Use \ct{..} in a list constant to define a list generator. List generators
build lists of numbers for you:

\begin{verbatim}
[1..10] == [1,2,3,4,5,6,7,8,9,10]
[1,3..10] == [1,3,5,7,9]
[10,9..1] == [10,9,8,7,6,5,4,3,2,1]
\end{verbatim}

\noindent
List generators are useful for expressing iteration. 

Lists may be infinite:

\begin{verbatim}
[1..] == [1,2,3,4,5,6,7,8,9 ..]
[5,4..] == [5,4,3,2,1,0,-1,-2,-3 ..]
\end{verbatim}

\noindent
Infinite lists are useful for expressing unbounded iteration.
See \pref{sec:lazy}.

You can write list comprehensions like this:

\begin{verbatim}
[x :: x <- [1..]; x % 2 == 0]
\end{verbatim}

\noindent
This could be read as {\em All x such that x is in \verb+[1..]+ and x is
even}, that is, the list of even numbers.

You can have any number of semicolon-separated qualifiers and each one can
be either a generator (like \verb"x <- [1..]") introducing a new variable
or pattern (see \pref{sec:pattern}), or a predicate (like \verb"x % 2 == 0") 
which filters the generators to the left of it.

Later generators change more rapidly, so for example:

\begin{verbatim}
[(x, y) :: 
  x <- [1..3]; y <- [x..3]] ==
    [(1, 1), (1, 2), (1, 3), 
      (2, 2), (2, 3), (3, 3)]
\end{verbatim}

You can nest list comprehensions to generate more complex data structures. For
example:

\begin{verbatim}
[[x * y :: x <- [1..10]] :: 
  y <- [1..10]]
\end{verbatim}

\noindent
will generate a times-table.

You can use pattern-matching (see \pref{sec:pattern}) to loop over several 
generators at the same time. For example:

\begin{verbatim}
[(x, y) :: [x, y] <- 
  zip2 [1..3] [1..3]] == 
    [(1, 1), (2, 2), (3, 3)]
\end{verbatim}

As a convenience, lists of characters may be written enclosed in double
quotes:

\begin{verbatim}
"abc" == ['a','b','c']
\end{verbatim}

You can define a string constant which has the same form as a variable name 
(that is, letters, numbers, underscore and apostrophy only) with a \verb+$+
prefix. For example:

\begin{verbatim}
$form7 == "form7"
\end{verbatim}

\noindent
\nip{} often uses these in option lists.

You can define a name, value pair with the \ct{=>} operator. 

\begin{verbatim}
$fred => 12 == ["fred", 12]
\end{verbatim}

\noindent
Again, these pairs are frequently used to pass options to objects.

A list may contain any object:

\begin{verbatim}
[1,'a',true,[1,2,3]] 
\end{verbatim}

\noindent
Mixing types in a list tends to be confusing and should be avoided. If you
want to group a set of diverse objects, define a class instead, 
see \pref{sec:class}.

Lists of lists of reals are useful for representing arrays.

You can use the list index operator (\ct{?}) to extract an element from a
position in a list:

\begin{verbatim}
[1,2,3] ? 0 == 1
"abc" ? 1 == 'b'
\end{verbatim}

You can use the list join operator (\ct{++}) to join two lists together
end-to-end.

\begin{verbatim}
[1,2,3] ++ [4,5,6] == [1,2,3,4,5,6]
\end{verbatim}

You can use the list difference operator (\verb+--+) to remove elements of one 
list from another.

\begin{verbatim}
[1..10] -- [4,5,6] == [1,2,3,7,8,9,10]
\end{verbatim}

\subsection{The function type}
\mylabel{sec:func}

Functions are objects just like any other. You can pass functions to other
functions as parameters, store functions in lists, and so on. 

You can create anonymous functions with \verb"\" (lambda). For example:

\begin{verbatim}
map (\x x + 2) [1..3] == [3, 4, 5]
\end{verbatim}

You can nest lambdas to make multi-argument anonymous functions, for example:

\begin{verbatim}
map2 (\x\y x + y) [1..3] [2..5] == 
  [3, 5, 7]
\end{verbatim}

You can compose functions with the \ct{@} operator. For example, for two
functions of one argument \ct{f} and \ct{g}:

\begin{verbatim}
f (g 2) == (f @ g) 2
\end{verbatim}

\subsection{The image type}

These represent a low-level handle to a VIPS image structure. You can make
them with the \ct{vips\_image} builtin, and you can pass them as parameters to
VIPS functions. The \ct{Image} class is built on top of them,
see \pref{sec:Image}.

As an accident of history, \nip{} also lets you do arithmetic with them. This
will probably be removed in the next version or two, so it's best to go
through the higher-level \ct{Image} class.

\section{Lists and recursion}
\mylabel{sec:lists}

Functional programming languages do not have variables, assignment or
iteration. You can achieve the same effects using just lists and recursion.

There are two main sorts of recursion over lists. The first is called 
\emph{mapping}: a function is applied to each element of a list, producing
a new list in which each element has been transformed.

\begin{verbatim}
map fn [a,b,c] == [fn a, fn b, fn c]
\end{verbatim}

The second main sort of recursion is called \emph{folding}: a list is turned
into a single value by joining pairs of elements together with a function and
a start value.

\begin{verbatim}
foldr fn start [a,b .. c] == 
  (fn a (fn b (.. (fn c start))))
\end{verbatim}

\noindent
(The function is called \ct{foldr} as it folds the list up right-to-left.
There is an analogous function called \ct{foldl} which folds a list up
left-to-right, but because of the way lists work, it is much slower and should
be avoided if possible.)

\ct{map} is defined in the standard list library for you:

\begin{verbatim}
/* map fn l: map function fn over list l
 */

map fn l
  = [], l == []
  = fn (hd l) : map fn (tl l) 
\end{verbatim}

\noindent
So, for example, you could use \ct{map} like this:

\begin{verbatim}
map (add 2) [1..5] == [3,4,5,6,7,8]
\end{verbatim}

\ct{foldr} is defined in the standard list library for you:

\begin{verbatim}
/* foldr fn st l: fold up list l, 
 * right to left with function fn and 
 * start value st
 */
 
foldr fn st l
  = st, l == []
  = fn (hd l) (foldr fn st (tl l)) 
\end{verbatim}

\noindent
So, for example, you could use \ct{foldr} like this:

\begin{verbatim}
foldr add 0 [1..5] == 15
\end{verbatim}

\noindent
(Mathematically, \ct{foldr} is the more basic operation. You can write
\ct{map} in terms of \ct{foldr}, but you can't write \ct{foldr} in terms of
\ct{map}.)

Unconstrained recursion over lists can be very hard to understand, rather like
\ct{goto} in an imperative language. It's much better to use a combination of
\ct{map} and \ct{foldr} if you possibly can.

The toolkit \ct{\_list} contains definitions of most of the standard
list-processing functions. These are listed in \tref{tb:list}. Check the
source for detailed comments.

\begin{tab2}
\begin{center}
\begin{tabular}{||l|l||}
\hline
Name & Description \\
\hline
\ct{all l} 		& and all the elements of list \ct{l} together \\
\ct{any l} 		& or all the elements of list \ct{l} together \\
\ct{concat l} 		& join a list of lists together \\
\ct{drop n l} 		& drop the first \ct{n} elements from list \ct{l} \\
\ct{dropwhile fn l} 	& drop while \ct{fn} is true \\
\ct{extract n l} 	& extract element \ct{n} from list \ct{l} \\
\ct{filter fn l} 	& all elements of \ct{l} for which \ct{fn} holds \\
\ct{foldl fn st l} 	& fold list \ct{l} left-to-right with \ct{fn} 
			  and \ct{st} \\
\ct{foldl1 fn l} 	& like \ct{foldl}, but use the first element of the 
			  list as the start value \\
\ct{foldr fn st l} 	& fold list \ct{l} right-to-left with \ct{fn} 
			  and \ct{st} \\
\ct{foldr1 fn l} 	& like \ct{foldr}, but use the first element of the 
			  list as the start value \\
\ct{index fn l} 	& search list \ct{l} for index of first element 
			  matching predicate \ct{fn} \\
\ct{init l} 		& remove last element of list \ct{l} \\
\ct{iterate f x} 	& repeatedly apply \ct{f} to \ct{x} \\
\ct{last l} 		& return the last element of list \ct{l} \\
\ct{len l} 		& find length of list \ct{l} \\
\ct{limit l}		& find the first element of list \ct{l} equal to
			  its predecessor \\
\ct{map fn l} 		& map function \ct{fn} over list \ct{l} \\
\ct{map2 fn l1 l2} 	& map 2-ary function \ct{fn} over lists \ct{l1}
		 	  and \ct{l2} \\
\ct{map3 fn l1 l2 l3} 	& map 3-ary function \ct{fn} over lists \ct{l1},
		 	  \ct{l2} and \ct{l3} \\
\ct{member l x}		& true if \ct{x} is a member of list \ct{l} \\
\ct{mkset eq l} 	& remove duplicates from list \ct{l} with equality 
			  function \ct{eq} \\
\ct{postfix l r} 	& add element \ct{r} to the end of list \ct{l} \\
\ct{product l} 		& product of list l \\
\ct{repeat x} 		& make an infinite list of \ct{x}es \\
\ct{replicate n x} 	& make \ct{n} copies of \ct{x} in a list \\
\ct{reverse l} 		& reverse list \ct{l} \\
\ct{scan fn st l} 	& apply \ct{(foldr fn r)} to every initial segment
			  of list \ct{l} \\
\ct{sort l} 		& sort list \ct{l} into ascending order \\
\ct{sortc fn l} 	& sort list \ct{l} into order by using a comparison
			  function \\
\ct{sortpl pl l} 	& sort list \ct{l} by predicate list \ct{pl} \\
\ct{sortr l} 		& sort list \ct{l} into descending order \\
\ct{split fn l} 	& break list \ct{l} into sections separated by
			  predicate \ct{fn} \\
\ct{splits fn l} 	& break list \ct{l} into single sections separated by
			  predicate \ct{fn} \\
\ct{splitpl pl l} 	& break list \ct{l} up by predicate list \ct{pl} \\
\ct{split\_lines n l} 	& break list \ct{l} into lines of length \ct{n} \\
\ct{sum l} 		& sum list l \\
\ct{take n l} 		& take the first \ct{n} elements from list \ct{l} \\
\ct{takewhile fn l} 	& take from the front of \ct{l} while \ct{fn} holds \\
\ct{zip2 l1 l2} 	& zip two lists together \\
\ct{zip3 l1 l2 l3} 	& zip three lists together \\
\hline
\end{tabular}
\end{center}
\caption{Functions in the standard list-processing toolkit}
\mylabel{tb:list}
\end{tab2}

\section{Lazy programming}
\mylabel{sec:lazy}

\nip{}'s programming language is \emph{lazy}, that is, it delays evaluation
as long as it possibly can.  For example, \ct{error} is a function which 
immediately halts execution of your function and pops up an alert window. So:

\begin{verbatim}
12 + error "wombat!"
\end{verbatim}

\noindent
Has no value: this expression will halt with an error message. However:

\begin{verbatim}
false && error "lasagne!"
\end{verbatim}

\noindent
Will evaluate to \ct{false}, since \nip{} knows after looking at the
left-hand-side of \ct{\&\&} that the result must be \ct{false}, and so does
not evaluate the right-hand-side.

\begin{verbatim}
[12, error "hot chilli!"] ? 0 == 12
\end{verbatim}

\noindent
This also evaluates completely, since the second element of the list is never
used, and therefore never evaluates.

Things become more confusing when you start calling functions, since the
arguments to a function call are also not evaluated until the function needs
that value. For example:

\begin{verbatim}
foldr (error "boink!") 2 [] == 2
\end{verbatim}

\noindent
Again, this evaluates successfully, since the function is never used by
\ct{foldr}.

\section{Pattern matching}
\mylabel{sec:pattern}

Any time you define a name, you can use a pattern instead. For example:

\begin{verbatim}
[fred, petra] = [12, 13]
\end{verbatim}

\noindent
defines \ct{fred} to have the value 12 and \ct{petra} to have the value 13.

A pattern describes the structure you are expecting for the value. When the
value is computed it is matched against the pattern and, if the match is
successful, the names in the pattern are bound to those parts of the value.
Our example is exactly equivalent to:

\begin{verbatim}
temp = [12, 13];
fred 
  = temp?0, is_list temp && 
      is_list_len 2 temp
  = error "pattern match failed";
petra 
  = temp?1, is_list temp && 
    is_list_len 2 temp
  = error "pattern match failed";
\end{verbatim}

\noindent
where \ct{temp} is an invisible, anonymous symbol.

You can pattern match on any of \nip{}'s data structures and types. You can
use:

\begin{description}

\item[\ct{a:b}]
Tests for the value being a non-empty list and then assigns \ct{a} to the
head and \ct{b} to the tail.

\item[\ct{(a,b)}]
Tests for the value being a complex and then assigns \ct{a} to the
real part and \ct{b} to the imaginary.

\item[\ct{[a,b,c]}]
Tests for the value being a list of length three and then assigns \ct{a},
\ct{b} and \ct{c} to the three elements.

\item[\ct{($class-name$ b)}]
Tests for the value being an instance of the named class, then assigns 
\ct{b} to that class instance.

\item[\ct{$constant$}]
Tests for the value being equal to that constant. Constants are things like
\ct{"hello world"} or \ct{12}.

\end{description}

You can nest patterns in any way you like. Patterns are useful in conjunction
with list comprehensions, see \pref{sec:listsyntax}.

You can't use patterns in function arguments in the current version, hopefully
this will added shortly.

\section{The standard libraries}

\nip{} comes with a lot of little utility functions. The functions for list
processing are listed in \tref{tb:list}. There are a huge number more, too
many to really list here. \tref{tb:toolkits} lists all the utility toolkits
with some hints about the kinds of function they contain. Read the (heavily
commented) toolkits for details.

\begin{tab2}
\begin{center}
\begin{tabular}{||l|l|l||}
\hline
Toolkit & Contains & Description \\
\hline
\ct{\_convert}		& \ct{parse\_int l}, \ldots{}  		& 
	convert ascii text to numbers \\
			& \ct{to\_matrix x}, \ldots{}  		& 
	convert anything into a matrix \\
			& \ct{colour\_transform\_to to x}, \ldots{}	& 
	convert between colour spaces \\
\hline
\ct{\_generate}		& \ct{image\_new w h ...}  		& 
	make a blank image \\
			& \ct{image\_white i}  			& 
	look at image \ct{i}, try to guess what white is \\
			& \ct{make\_xy w h}  			& 
	make an image of size \ct{w} by \ct{h} whose pixel value are \\
			& 		  			& 
	their coordinates \\
\hline
\ct{\_types}		& \ct{Image i}  			& 
	all the standard classes and support functions, \\
			& 		  			& 
	see \pref{sec:object} \\
\hline
\ct{\_predicate}	& \ct{is\_colour\_space i} 		& 
	test for objects are in various categories or have \\
			& 		  			& 
	various properties \\
\hline
\ct{\_stdenv}		& \ct{logical\_and x}, \ldots{}		& 
	function versions of all the operators \\
			& \ct{bandsplit i}, \ldots{}		& 
	break up and recombine images by band \\
			& \ct{mean x}, \ldots{}			& 
	statistical ops on objects \\
			& \ct{transpose x}, \ct{flipud x}, \ct{rot90 x},
			\ldots{} 				& 
	flips, rotates, etc. on objects \\
			& \ct{rad x}, \ct{pi}, \ldots{}		& 
	trigonometry stuff \\
			& \ct{sign x}, \ct{conj x}, \ct{polar x}, \ldots{} & 
	complex stuff \\
			& \ct{rint x}, \ct{ceil x}, \ldots{}	& 
	various rounding things \\
			& \ct{fwfft x}, \ldots{}		& 
	fourier stuff \\
			& \ct{dilate m x}, \ct{rank w h n i}, \ldots{} & 
	morphology stuff \\
			& \ct{conv m x}, \ldots{}		& 
	convolution stuff \\
			& \ct{image\_set\_type t i}, \ldots{}  	& 
	set various image header field \\
			& \ct{resize x y i}, \ldots{}		& 
	resampling images \\
			& \ct{recomb m i}, \ldots{}		& 
	recombinations \\
			& \ct{clip2fmt f i}, \ldots{}		& 
	format conversions \\
			& \ct{hist\_find m x}, \ldots{}		& 
	histogram stuff \\
			& \ct{id x}, \ct{const x y}, \ldots{} 	& 
	various useful operations on functions \\
			& \ct{map\_binary fn x y},  \ldots{} 	& 
	mapping over groups \\
\hline
\end{tabular}
\end{center}
\caption{Useful utility functions --- see the source for details}
\mylabel{tb:toolkits}
\end{tab2}

\section{Classes}
\mylabel{sec:class}

You can define new types using \ct{class}. For example:

\begin{verbatim}
Pasta_plain = class {
  lasagne = "large sheets";
  fusilli = "sort of twisty";
  radiatori = "lots of ridges";
}
\end{verbatim}

\noindent
This defines a new class called \ct{Pasta\_plain}. The class has three members
(\ct{lasagne}, \ct{fusilli} and \ct{radiatori}), each of which has a list of
\ct{char} as its value. By convention, we've named classes with an initial
capital letter, but of course you can do what you like.

You can refer to the members of a class using the class project (\ct{.}) 
operator. For example:

\begin{verbatim}
Pasta_plain.lasagne == "large sheets"
\end{verbatim}

\noindent
You can use an expression to the right of \ct{.} if you enclose it in
brackets. For example:

\begin{verbatim}
Pasta_plain.("las" ++ "agne") == 
  "large sheets"
\end{verbatim}

Classes can contain any objects as members, including functions and
sub-classes. Functions may define local classes, classes may define local
functions, and all may refer to each other using the usual scope rules. For
example:

\begin{verbatim}
Pasta_all = class { 
  filled = class { 
    tortelloni = "venus' navel";
    ravioli = "square guys";
  }
  plain = Pasta_plain;
}
\end{verbatim}

When you define a class, \nip{} adds a few extra members for you. \ct{name}
is a list of \ct{char} giving the name of the class. \ct{this} and
\ct{super} are the most-enclosing class instance and the class instance
this class is derived from (see \pref{sec:inheritance}). \nip{} also adds
a default constructor: a member with the same name as the class, pointing
back to the class constructor. 

For efficiency reasons \nip{} does not allow mutual recursion at the top
level. If two functions depend on each other, neither will ever be
calculated. For example:

\begin{verbatim}
a = 1 : b;
b = 2 : a;
\end{verbatim}

\noindent
Neither \ct{a} nor \ct{b} will have a value.

You can have mutual recursion between class members. For example:

\begin{verbatim}
Fred = class {
  a = 1 : b;
  b = 2 : a;
}
\end{verbatim}

\noindent
Now \ct{Fred.a} will have the value \ct{[1, 2, 1, 2, 1, \ldots{}]}.

\subsection{Parameterised classes}

Classes can have parameters. Parameters behave like class members initialised
from arguments to the class constructor. For example:

\begin{verbatim}
My_pasta pasta_name cooked = class {
  is_ready t = "your " ++ 
    pasta_name ++ " is " ++ state
  {
    state
      = "underdone!", t < cooked
      = "perfect", t == cooked
      = "yuk!";
  }
}
\end{verbatim}

\noindent
This defines a class called \ct{My\_pasta} which takes a pasta name
and a cooking time as parameters. Once you have made an instance of
\ct{My\_pasta}, you can test if it's been cooked at a certain time with the
\ct{is\_ready} member. For example:

\begin{verbatim}
tele = My_pasta "telephoni" 10;
tele.is_ready 5 == 
  "your telephoni is underdone!"
\end{verbatim}

\subsection{Inheritance}
\mylabel{sec:inheritance}

Classes can inherit from a super-class. For example:

\begin{verbatim}
Pasta_more = class Pasta_plain { 
  macaroni = "tubes";
  spaghetti = "long and thin";
  lasagne = "fairly large sheets";
}
\end{verbatim}

\noindent
Here the new class \ct{Pasta\_more} inherits members from the previous class
\ct{Pasta\_plain}. It also overrides the definition of \ct{lasagne} from
\ct{Pasta\_plain} with a new value. For example:

\begin{verbatim}
Pasta_more.macaroni == "tubes"
Pasta_more.fusilli == "sort of twisty"
Pasta_more.lasagne == "fairly large sheets"
\end{verbatim}

You can use \ct{this} and
\ct{super} to refer to other members up and down the class hierarchy.
\ct{super} is the class instance that the current class inherits from (if
there's no super-class, \ct{super} has the value \ct{[]}), and
\ct{this} is the most-enclosing class instance. 

\begin{verbatim}
Pasta_more.super == Pasta_plain
Pasta_more.this == Pasta_more
Pasta_more.super.this == Pasta_more 
\end{verbatim}

\noindent
therefore:

\begin{verbatim}
Pasta_more.lasagne == "fairly large sheets"
Pasta_more.super.lasagne == "large sheets"
Pasta_more.super.this.lasagne ==
  "fairly large sheets" 
\end{verbatim}

There's a special symbol \ct{root} which encloses all symbols. For
example:

\begin{verbatim}
fred = 12; 

Freddage = class { 
  fred = 42;
  mystery = root.fred;
}
\end{verbatim}

\noindent
Now \ct{Fred.mystery} will have the value 12.

There's another special symbol called \ct{scope} which encloses all symbols in
the file this definition was loaded from. If you want to refer to another
definition in the same file which is being masked somehow, use \ct{scope}.

You can use the built in function \ct{is\_instanceof} to test whether an
instance is or inherits from a class. For example:

\begin{verbatim}
is_instanceof "Pasta_more" Pasta_more == true
is_instanceof "Pasta_plain" Pasta_more == true
is_instanceof "Pasta_more" Pasta_plain ==
  false
\end{verbatim}

The super-class constructor can take arguments, and these arguments can refer
to class members. For example:

\begin{verbatim}
Fresh_pasta pasta_name = class
  My_pasta pasta_name cooked {
  cooked = 2;
}
\end{verbatim}

\noindent
Defines a class for fresh pasta, which always cooks in 2 minutes. You need to
be careful not to make loops: if \ct{cooked} did tried to refer to
something in the super-class, this class would never construct properly.
\nip{} unfortunately does not check for this error.

Finally, the superclass can be a fully constructed class. In this case, the
superclass is cloned and the new class members wrapped around it. You can use
this to write a class which can wrap any other class and add members to it.
Many of the toolkit menu items use this trick to enable them to work for any
object type.

\subsection{Minor class features}

There are a couple of other things you can do with classes. You can define a
special member called \ct{\_check}. If this member is defined, then when a
class instance is created, the check member is returned instead of the class
itself. You can use this to implement class argument type checks, for example:

\begin{verbatim}
Fred a b = class { 
  _check
    = this, is_real a && is_real b
    = error "args to Fred must " ++
      "both be real"
}
\end{verbatim}

\noindent
Defines a class called \ct{Fred} which has to have two real numbers as
arguments.

You can define members called \ct{oo\_binary}, \ct{oo\_binary'} and
\ct{oo\_unary} and do operator overloading. When \nip{} sees one of the
standard operators being used on an instance of your class, it will look up
one of these members and pass in the name of the operator and the argument.
The two forms of the binary operator member are called for the class-on-left
and the class-on-rights cases. So:

\begin{verbatim}
x = Fred 1 2
x + 12 == x.oo_binary "add" 12
12 + x == x.oo_binary' "add" 12
!x == x.oo_unary "negate" 
\end{verbatim}

These two features are very primitive. The \ct{\_Object} class in the
\ct{\_types} toolkit builds on these to provide a fairly high-level system for
checking class arguments and defining the meaning of operators.
See \pref{sec:object}.

\section{Controlling the interface}
\mylabel{sec:bowser}

\nip{} looks at the scraps of program you type in and execute and tries to
show them on the screen in a graphical way. The sorts of display you get
depend on where in \nip{} you define the expression, and what sort of value it
has.

\subsection{Tools and toolkits}
\mylabel{sec:tools}

Definitions in toolkits are turned into menus off the \ct{Toolkits} menu
in the main window, and added to the toolkit browser. Toolkits are loaded
from files at startup or can be made in the program window.
Toolkit or a definition names which start with
an underscore character are hidden and not displayed. The toolkits are always
displayed in alphabetical order, but you can order the items within a toolkit
in any way you like. 

There are two ways to write toolkit definitions. Function definitions and
zero-argument classes simply appear as menu items, built from static analysis
of their source code. However, if a definition evaluates to an instance of the
class \ct{Menu}, a menu item is built from dynamic analysis of the value of
the definition.

\subsubsection{Static menu items}

Zero-argument classes within toolkits are displayed
as pull-right menus. You can nest classes to any depth. 

\nip{} uses the first line of the comment before a definition as
help text for that function, so it's a good idea to put a simple one-line
description of the function at the start of a comment.

For example, if the following text is placed in a file called \ct{Fred.def}
on \nip{}'s start path, you'll get a menu in the tookits called \ct{Fred} with
a pull-right and a tooltip. See \fref{fg:toolkit}.

\begin{verbatim}
Banana a = a * 3;

Subfred = class { 
  // add two things
  Jim a b = a + b;
  Apple e = e * 12;
  Harry z = 12 + z;
}
\end{verbatim}

\begin{figure}
\figw{2.5in}{toolkit.jpg}
\caption{How \ct{Fred.def} will look}
\mylabel{fg:toolkit}
\end{figure}

\subsubsection{Dynamic menu items}

Dynamic menus give you much more control over the way menus are drawn and make
it easy to reuse menus. A dynamic menu item is a class instance that is a
sub-class of \ct{Menuitem}. It needs to have three members: \ct{label}, the
text that should appear in the menu (with an underscore character to indicate
the mnenonic); \ct{tooltip}, a short hint that appears as a tooltip or in
the toolkit browser; \ct{icon}, an optional image file to be displayed in the 
menu next to the text; and \ct{action}, the function that is called when
the menu item is activated. \ct{label} and \ct{tooltip} are constructor
arguments for \ct{Menu}.

So for example:

\begin{verbatim}
Wombat_find_item = class Menuitem
  "_Find Wombat"
  "analyse image and locate wombat" {
  icon = "nip-slider-16.png";
  action x = im_wombat_locate x;
}
\end{verbatim}

\noindent
will appear as shown in \fref{fg:toolkit2}.

\begin{figure}
\figw{2.5in}{toolkit2.jpg}
\caption{How \ct{Wombat\_find\_item} will look}
\mylabel{fg:toolkit2}
\end{figure}

A dynamic pullright menu is a subclass of \ct{Menupullright}. It's just like
\ct{Menuitem}, but without the need for an \ct{action} member. Any members
which are subclasses of \ct{Menu} are displayed as items in the submenu. So
again:

\begin{verbatim}
Wombat_item = class Menupullright
  "_Wombat"
  "wombat-related operations" {
  icon = "nip-slider-16.png";
  item1 = Wombat_find_item;
  sep = Menuseparator;
  boink = Wombat_find_item;
}
\end{verbatim}

\noindent
will appear as shown in \fref{fg:toolkit3}.

\begin{figure}
\figw{2.5in}{toolkit3.jpg}
\caption{How \ct{Wombat\_item} will look}
\mylabel{fg:toolkit3}
\end{figure}

\subsection{Workspaces}
\mylabel{sec:workspaces}

Definitions in workspaces are displayed with \nip{}'s class browser. Each row
is displayed in four main parts: a button for the row name, a line of text,
a set of sub-rows for the members of the row's class, and a graphic display
representing the row's value. See \fref{fg:row2}.

\begin{figure}
\figw{2.5in}{ir8a.jpg}
\caption{Components of a workspace row}
\mylabel{fg:row2}
\end{figure}

The text part of the right-hand-side of each row is always displayed, but the
sub-rows are only displayed if the row represents a class, and the graphic is
only displayed if the class is an instance of one of the classes in
\tref{tb:classes}. You can subclass these if you want to use the graphic
display in your own widgets.

There are three separate ways to set the value for a row. You can
edit the line of program text, you can edit one of the members, or you can
manipulate the graphic representation (dragging a slider, or moving a region).
These can be contradictory, so \nip{} resolves conflicts by always applying
changes in the order text, then graphic, then member. 

When it applies a graphic change, \nip{} rebuilds the class using a class 
member called \emph{class-name}\ct{\_edit}, or if that is not defined, the
class's constructor member. For example, the \ct{Colour} class can be defined
as:

\begin{verbatim}
Colour colour_space value = class {}
A1 = Colour "sRGB" [255,0,0];
\end{verbatim}

\noindent
There are two ways to change \ct{A1}. You can open \ct{A1} and change
\ct{colour\_space} to \ct{"Lab"}, or you can double-click on the swatch
and drag the disc. When you click \ct{OK} on the colour edit dialog, \nip{}
searches for a member called \ct{Colour\_edit}, fails to find it, and so
picks the \ct{Colour} member instead (the default constructor generated by
\nip{}). It then replaces the value of A1 with

[needs finishing]

\begin{tab2}
\begin{center}
\begin{tabular}{||l|l||}
\hline
Class & Description \\
\hline
\ct{Clock \emph{interval} \emph{value}}				& 
	A clock widget, handy for animations \\
\ct{Expression \emph{caption} \emph{expr}}			& 
	Displays an editable expression \\
\ct{Group \emph{value}}						& 
	A group of objects for iteration \\
\ct{List \emph{value}}						& 
	A list of related objects \\
\ct{Pathname \emph{caption} \emph{value}}			& 
	Displays a file browser \\
\ct{Fontname \emph{caption} \emph{value}}			& 
	Displays a font browser \\
\ct{Toggle \emph{caption} \emph{value}}				& 
	A toggle switch \\
\ct{Scale \emph{caption} \emph{from} \emph{to} \emph{value}}	& 
	A slider \\
\ct{Option \emph{caption} \emph{labels} \emph{value}}		& 
	Select one item from a list \\
\ct{Colour \emph{colour\_space} \emph{value}}			& 
	A patch of colour \\
\ct{Matrix\_vips \emph{value} \emph{scale} \emph{offset} \emph{filename} 
	\emph{display}}						& 
	A matrix \\
\ct{Arrow \emph{image} \emph{left} \emph{top} \emph{width} \emph{height}} & 
	Two points joined by a line on an image \\
\ct{Region \emph{image} \emph{left} \emph{top} \emph{width} \emph{height}} & 
	A sub-area of an image \\
\ct{Plot \emph{options} \emph{value}}				& 
	Displays a plot widget \\
\ct{Image \emph{value}}						& 
	An image \\
\ct{Number \emph{caption} \emph{value}}				& 
	Displays an editable number \\
\ct{Real \emph{value}}						& 
	Displays a real number \\
\ct{Vector \emph{value}}					& 
	Displays a list of reals \\
\ct{String \emph{caption} \emph{value}}				& 
	Displays an editable string \\
\ct{Mark \emph{image} \emph{left} \emph{top}} 			& 
	A point on an image \\
\ct{HGuide \emph{image} \emph{top}} 				& 
	A horizontal line on an image \\
\ct{VGuide \emph{image} \emph{left}} 				& 
	A vertical line on an image \\
\ct{Area \emph{image} \emph{left} \emph{top} \emph{width} \emph{height}} & 
	A sub-area of an image, fixed in size \\
\hline
\end{tabular}
\end{center}
\caption{\nip{} built in graphic classes}
\mylabel{tb:classes}
\end{tab2}

\subsection{The \ct{Image} class}
\mylabel{sec:Image}.

say supports mixed ops with real, vector and complex constants 

\subsection{The \ct{Colour} class}
\mylabel{sec:colour}

This class displays a swatch of colour. If you double-click on the

\begin{verbatim}
Pathname caption value = class {}
\end{verbatim}

\section{The \ct{\_Object} class}
\mylabel{sec:object}

\section{Optimisation}
\mylabel{sec:optimise}

\nip{} performs three useful optimisations on expressions. First, it finds and
removes common sub-expressions in functions. So for example:

\begin{verbatim}
if a + b < 12 then a + b else b
\end{verbatim}

\noindent
will only evaluate \ct{a + b} once. This can save a lot of time if \ct{a} or
\ct{b} is a large image. 

Second, \nip{} detects arithmetic operations on \ct{unsigned char} images, and
replaces them with look-up tables. For example:

\begin{verbatim}
a = vips_image "campin.v"
b = a * (a - 1) ** 0.5
\end{verbatim}

\noindent
Provided \ct{campin.v} is an 8 bit image image, this expression will
evaluate with a single call to \ct{im\_maplut()}.

Finally, \nip{} has a VIPS operation cache. It memorises the arguments to the 
last few hundred calls to VIPS, and the result each call gave. Before calling
VIPS again, it checks to see if there is a previous call with the same
arguments and if there is, uses the result it obtained last time.

\section{Calling VIPS functions}
\mylabel{sec:callvips}

You can call any VIPS operation which has the following properties:

\begin{itemize}

\item
There must be at least 1 output argument. If there's a single output argument,
that becomes the value of the function. If there is more than one output, then
the function returns a list with the outputs as members.

\item
The output arguments must all be one of:

\begin{itemize}
\item \verb+IM_TYPE_DOUBLE+, 
\item \verb+IM_TYPE_INT+, 
\item \verb+IM_TYPE_COMPLEX+, 
\item \verb+IM_TYPE_STRING+, 
\item \verb+IM_TYPE_IMAGE+, 
\item \verb+IM_TYPE_DOUBLEVEC+, 
\item \verb+IM_TYPE_DMASK+, 
\item \verb+IM_TYPE_IMASK+
\end{itemize}

\item
The input arguments must all be one of the types above, or
\verb+IM_TYPE_DISPLAY+. If an argument is an input display, \nip{} passes in
its current display structure, it does not take a display from your program.

\end{itemize}

When \nip{} starts up, it loads any VIPS plug ins it can find on its data
search path. You can call functions from plug ins in just the same way. For
information on writing plug ins, see the \emph{VIPS Manual}.

